\section{Methods}\label{sec:methods}

\subsection{Dataset and Study Design}\label{subsec:dataset}

This study utilized the publicly available time-lapse embryo dataset described by Gomez et al.~\cite{gomez2022timelapse}, comprising 704 time-lapse videos of developing embryos with 2.4M images across 7 focal planes. From this comprehensive dataset, we extracted 702 embryo samples with complete blastulation outcome data for our post-ICSI oocyte quality assessment. Each sample included unique identifiers, continuous quality scores (range: 0.004-0.999, mean: 0.592 ± 0.287), and binary blastulation labels (66.7\% successful outcomes, reflecting typical clinical IVF success rates)~\cite{awadalla2021age,zhu2024developmental}. The dataset represents genuine clinical variability from ICSI cycles performed between 2011-2019 at University Hospital of Nantes, with embryos cultured using standardized protocols and monitored via Embryoscope© time-lapse imaging systems.

All data analysis was performed using real clinical outcomes to ensure clinically relevant performance metrics~\cite{varoquaux2022machine}. Cross-validation was implemented using 8-fold stratified sampling to maintain class balance and provide robust performance estimates across the complete dataset~\cite{hastie2009elements}.

\subsection{Parametric Cycle Prediction Calculator}\label{subsec:calculator}

The parametric calculator incorporates established clinical relationships through transparent mathematical models, enabling interpretable predictions suitable for patient counseling~\cite{rudin2019stop}.

\subsubsection{Age-Dependent AMH Modeling}

The calculator implements age-specific AMH percentile distributions derived from published reproductive medicine literature~\cite{lee2017amh,song2021amh}. Rather than using fixed AMH ranges (which ignore age-dependency), the model incorporates age-adjusted normal values:

\begin{equation}
\text{AMH}_{\text{normal}}(\text{age}) = f(\text{age-specific percentiles})
\end{equation}

where $f$ represents a sigmoid function parametrized from published age-stratified AMH distributions~\cite{lee2017amh}. The function captures the dramatic age-related decline in AMH percentiles: median values decrease from $\approx$ 1.8 ng/mL at age 25 to $\approx$ 0.18 ng/mL at age 42, with 10th and 90th percentiles showing parallel decline patterns~\cite{lee2017amh}.

\subsubsection{Comprehensive IVF Cycle Prediction Pipeline}

The parametric model implements a complete IVF cycle simulation from oocyte retrieval through live birth, incorporating stage-specific attrition rates and patient-specific factors~\cite{seifer2002amh,ovarian_reserve_testing}.

\textbf{Oocyte Retrieval Prediction:}
The initial oocyte yield combines age effects, AMH adjustments, and optional antral follicle count (AFC) data through established clinical relationships:

\begin{equation}
\text{Retrieved Oocytes} = \text{Base}(\text{age}) \times \text{AMH Factor}(\text{AMH}, \text{age}) \times \text{Health Factors}
\end{equation}

The age-dependent baseline utilizes a sigmoid function to capture the nonlinear decline in ovarian response with advancing age~\cite{acog2017advanced}:
\begin{equation}
\text{Base}(\text{age}) = \text{sigmoid}(\text{age}, \text{age-specific parameters})
\end{equation}

The AMH factor employs a Gompertz growth function to model the relationship between AMH ratios and oocyte yield multipliers:
\begin{equation}
\text{AMH Factor} = \text{Gompertz}\left(\frac{\text{AMH}_{\text{patient}}}{\text{AMH}_{\text{normal}}(\text{age})}\right)
\end{equation}

When AFC data is available, the model incorporates the Ovarian Response Prediction Index (ORPI) for enhanced accuracy:
\begin{equation}
\text{ORPI} = \frac{\text{AMH} \times \text{AFC}}{\text{age}}
\end{equation}

\textbf{Complete Attrition Pipeline:}
The model simulates the full IVF process through age-stratified attrition rates at each critical stage:

\begin{align}
\text{Frozen} &= f_{\text{freeze}}(\text{Retrieved}, \text{age}, \text{ORPI}) \\
\text{Thawed} &= \text{Frozen} \times \alpha_{\text{thaw}}(\text{age}) \\
\text{Fertilized} &= \text{Thawed} \times \alpha_{\text{fert}}(\text{age}) \\
\text{Good Embryos} &= \text{Fertilized} \times \alpha_{\text{embryo}}(\text{age}) \\
\text{Implanted} &= \text{Good Embryos} \times \alpha_{\text{implant}}(\text{age}) \times \text{Patient Factors} \\
\text{Live Birth} &= \text{Implanted} \times \alpha_{\text{birth}}
\end{align}

where $\alpha_{\text{stage}}(\text{age})$ represents age-specific attrition rates derived from clinical outcomes, and Patient Factors incorporate BMI effects (polynomial function), ethnicity adjustments, and health condition modifiers (e.g., PCOS enhances oocyte yield by 20\%, endometriosis reduces implantation by 20\%) based on published clinical evidence~\cite{lee2017amh}.

\textbf{Multiple Cycle Projections:}
The framework provides predictions for up to three consecutive cycles, accounting for age progression and cumulative live birth probabilities using established probability theory for independent trials with age-dependent success rates.

This comprehensive approach employs biologically-motivated nonlinear functions and stage-specific attrition modeling that accurately captures the complex relationships throughout the entire IVF process, while maintaining clinical interpretability through transparent mathematical formulations that can be explained to patients during counseling sessions.

\subsection{Oocyte Quality Assessment Model}\label{subsec:quality}

\subsubsection{Vision Transformer Architecture}

We implemented a Vision Transformer (ViT) model for oocyte quality assessment~\cite{dosovitskiy2021image}, leveraging the architecture's demonstrated effectiveness in medical image analysis~\cite{alhammuri2023vision}. The model processes standardized post-ICSI oocyte images (224×224 pixels, acquired immediately after intracytoplasmic sperm injection but before pronuclear formation) through attention mechanisms that can capture subtle morphological features relevant to blastulation potential~\cite{zhang2021machine}.

The ViT architecture consists of:
\begin{itemize}
\item Patch embedding layers (16×16 patches) for image tokenization into 196 patches
\item 12-layer transformer with multi-head self-attention mechanisms (12 heads)
\item Position encoding for spatial relationship preservation
\item MLP classification head with dropout (0.1) for blastulation prediction
\item Model parameters: ViT-Base/16 configuration with 86M parameters
\end{itemize}

\textbf{Image Acquisition Protocol:} Post-ICSI oocyte images were acquired using the Embryoscope© time-lapse incubator system (Vitrolife©, Sweden) with a camera under a 635 nm LED light source passing through Hoffman's contrast modulation optics~\cite{gomez2022timelapse}. Images were captured every 10-20 minutes from fertilization through blastocyst development, with our analysis focusing on post-ICSI, pre-2PN timepoints. Original images were normalized for brightness and contrast, then resized to 224×224 pixels while maintaining aspect ratio through center cropping.

\subsubsection{Training and Validation}

Model training utilized the complete 702-sample dataset with 8-fold cross-validation~\cite{varoquaux2022machine}. Each fold maintained stratified sampling to preserve the 66.7\% positive class distribution observed in clinical practice.

Training parameters were optimized for reproducibility and performance:

\begin{center}
\small
\begin{tabular}{ll}
\hline
\textbf{Parameter} & \textbf{Value} \\
\hline
Optimizer & AdamW \\
Initial Learning Rate & 3e-4 \\
Weight Decay & 0.01 \\
Beta Values & (0.9, 0.999) \\
Batch Size & 16 \\
Max Epochs & 100 \\
Learning Rate Schedule & Cosine annealing with warm-up (5 epochs) \\
Early Stopping Patience & 10 epochs \\
Early Stopping Metric & Validation AUC \\
Minimum Delta & 0.001 \\
Dropout Rate & 0.1 \\
Data Augmentation & Random horizontal flip (p=0.5), rotation ($\pm$15°) \\
Normalization & Z-score (channel-wise) \\
\hline
\end{tabular}
\end{center}

\textbf{Table 2:} Vision Transformer Training Hyperparameters

\subsubsection{Performance Evaluation}

Model performance was assessed using multiple metrics appropriate for clinical applications~\cite{litjens2017survey}:

\textbf{Continuous Prediction Metrics:}
\begin{itemize}
\item Pearson correlation coefficient (r)
\item Mean Absolute Error (MAE) 
\item Root Mean Square Error (RMSE)
\end{itemize}

\textbf{Binary Classification Metrics:}
\begin{itemize}
\item Accuracy, Precision, Recall, F1-Score
\item Area Under the ROC Curve (AUC)
\item Sensitivity and Specificity
\item Positive and Negative Predictive Values
\end{itemize}

\textbf{Statistical Validation:}
Cross-validation error bars were computed across all folds to quantify model stability. Statistical significance testing used Mann-Whitney U tests~\cite{mann1947test} comparing model performance against multiple baselines: (1) label-shuffled controls, (2) majority class classifier, and (3) random prediction baseline. Effect size calculations used Cohen's d~\cite{cohen1988statistical}. 

\textbf{Baseline Comparisons:} Model performance was evaluated against established baselines:
\begin{itemize}
\item \textbf{Random Classifier}: AUC = 0.500, accuracy = 50\%
\item \textbf{Majority Class}: Always predicts positive class (66.7\% accuracy)
\item \textbf{Label Shuffle}: Same model architecture trained on randomized labels
\item \textbf{Morphological Scoring}: Traditional embryologist assessment metrics
\end{itemize}

\subsection{Integrated Framework Implementation}

The dual-model framework was designed for clinical integration~\cite{fda2022clinical}, providing:

\begin{enumerate}
\item \textbf{Parametric Calculator Interface}: Web-based tool allowing clinicians to input patient age and AMH values for immediate oocyte yield predictions with transparent assumptions.

\item \textbf{Quality Assessment Pipeline}: Automated processing of oocyte images through the trained ViT model, providing quality scores and blastulation probability estimates.

\item \textbf{Combined Reporting}: Integrated output combining cycle predictions with individual oocyte quality assessments for comprehensive patient counseling~\cite{asrm2021counselors}.
\end{enumerate}

All implementations prioritized transparency and clinical interpretability over opaque model accuracy~\cite{topol2019high}, ensuring that predictions could be meaningfully discussed with patients and incorporated into clinical decision-making processes~\cite{beauchamp2019principles}. 