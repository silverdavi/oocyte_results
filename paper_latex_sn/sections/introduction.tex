\section{Introduction}\label{sec:introduction}

In vitro fertilization (IVF) represents a critical intervention for couples facing infertility~\cite{hfea2024statistics,sart2024national}, yet current counseling approaches rely heavily on population-based statistics that inadequately capture individual patient variability~\cite{gameiro2023understanding}. Traditional IVF counseling typically provides aggregate success rates stratified by broad demographic categories, failing to account for the complex interplay between patient-specific factors such as age, ovarian reserve markers, and individual oocyte quality characteristics~\cite{asrm2017embryo}.

The limitations of current approaches are particularly evident in two key areas: cycle outcome prediction and oocyte quality assessment. For cycle predictions, clinicians often rely on simplified age-based estimates that ignore the substantial individual variation in ovarian reserve as measured by anti-Müllerian hormone (AMH) levels~\cite{seifer2002amh,ovarian_reserve_testing}. These population averages can mislead patients about their specific prognosis, particularly given the dramatic age-dependent changes in AMH percentiles that are rarely considered in counseling protocols~\cite{lee2017amh,song2021amh}.

For oocyte quality assessment, current practice relies primarily on morphological evaluation by embryologists—a subjective process with significant inter-observer variability and limited predictive accuracy~\cite{paternot2009observer,paternot2011multicentre,fordham2022embryologist}. Recent studies have demonstrated that even experienced embryologists show only moderate agreement when assessing blastocyst implantation probability, with artificial intelligence models often outperforming human assessment~\cite{fordham2022embryologist}. While blastulation rates provide some indication of developmental competence~\cite{zhu2024developmental}, the assessment typically occurs after critical treatment decisions have already been made. This reactive approach limits opportunities for personalized treatment optimization and informed patient counseling~\cite{racowsky2010standardization}.

Recent advances in artificial intelligence and machine learning offer promising avenues for improving IVF counseling through more personalized, data-driven approaches~\cite{litjens2017survey,zhang2021machine}. Vision Transformer (ViT) models have demonstrated remarkable capabilities in medical image analysis~\cite{dosovitskiy2021image,alhammuri2023vision}, while parametric modeling approaches can incorporate established clinical relationships in transparent, interpretable frameworks~\cite{rudin2019stop}. Previous work has shown that algorithmic approaches can significantly enhance predictive accuracy in embryo implantation assessment~\cite{rave2024bonna}, demonstrating the clinical potential of AI-assisted reproductive medicine.

In this work, we present a comprehensive dual-model framework that addresses both cycle prediction and oocyte quality assessment challenges. Our approach combines: (1) a parametric calculator that incorporates age-dependent AMH percentiles and established clinical relationships to provide transparent, interpretable predictions of oocyte retrieval yields, and (2) a Vision Transformer model trained on real post-ICSI oocyte images (pre-2PN) to assess individual oocyte quality and predict blastulation outcomes.

This integrated framework is designed to provide clinicians with evidence-based tools for personalized IVF counseling while maintaining realistic expectations about model performance and clinical applicability~\cite{asrm2021counselors}. Rather than claiming unrealistic accuracy, our approach acknowledges the inherent limitations of prediction in reproductive medicine while demonstrating meaningful improvements over current subjective assessment methods~\cite{varoquaux2022machine}. 