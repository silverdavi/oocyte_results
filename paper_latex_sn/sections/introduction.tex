\section{Introduction}\label{sec:introduction}

Assisted reproductive technology (ART) represents a critical intervention for couples facing infertility and individuals seeking fertility preservation~\cite{hfea2024statistics,sart2024national}, yet current counseling approaches rely heavily on population-based statistics that inadequately capture individual patient variability~\cite{gameiro2023understanding}. Traditional protocols provide aggregate success rates stratified by broad demographics, failing to account for patient-specific factors such as age-dependent ovarian reserve markers and individual oocyte quality characteristics~\cite{asrm2017embryo}.

Current limitations are particularly evident in cycle outcome prediction and oocyte quality assessment. For predictions, clinicians often rely on simplified age-based estimates that ignore substantial individual variation in anti-Müllerian hormone (AMH) levels~\cite{seifer2002amh,ovarian_reserve_testing}. The dramatic age-dependent changes in AMH percentiles—declining from $\sim$1.8 ng/mL at age 25 to $\sim$0.18 ng/mL at age 42—are rarely considered in counseling protocols, leading to misleading patient prognoses~\cite{lee2017amh,song2021amh}.

For oocyte assessment, current practice relies on subjective morphological evaluation with significant inter-observer variability and limited predictive accuracy~\cite{paternot2009observer,paternot2011multicentre,fordham2022embryologist}. Even experienced embryologists show only moderate agreement when assessing blastocyst implantation probability, with artificial intelligence models often outperforming human assessment~\cite{fordham2022embryologist}. This reactive approach limits opportunities for personalized treatment optimization~\cite{racowsky2010standardization}.

Recent advances in artificial intelligence offer promising solutions through Vision Transformer (ViT) models that have demonstrated remarkable capabilities in medical image analysis~\cite{dosovitskiy2021image,alhammuri2023vision}, while parametric modeling can incorporate established clinical relationships in transparent frameworks~\cite{rudin2019stop}. Previous work has shown algorithmic approaches can significantly enhance predictive accuracy in embryo assessment~\cite{rave2024bonna,silver2020datadriven}, demonstrating clinical potential for AI-assisted reproductive medicine. The broader adoption of "deep technology" solutions across IVF laboratories—including automated cryostorage systems, digital inventory management, and integrated data analytics—reflects the field's commitment to technological advancement~\cite{go2023deep}.

We present a comprehensive dual-model framework addressing both challenges through: (1) a parametric calculator incorporating age-dependent AMH percentiles and established clinical relationships for transparent oocyte retrieval predictions, and (2) a Vision Transformer model trained on post-ICSI oocyte images to assess individual quality and predict blastulation outcomes.

This framework provides clinicians with evidence-based tools for personalized IVF counseling while maintaining realistic performance expectations~\cite{asrm2021counselors}. Our approach exemplifies responsible AI development, prioritizing transparency and honest performance reporting over unrealistic accuracy claims~\cite{varoquaux2022machine}, demonstrating meaningful improvements over current subjective methods through rigorous validation and uncertainty quantification. 