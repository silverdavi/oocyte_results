\section{Conclusion}\label{sec:conclusion}

We have developed and validated a comprehensive dual-model framework for personalized IVF counseling that addresses critical limitations in current clinical practice~\cite{gameiro2023understanding,asrm2017embryo}. The integration of parametric cycle prediction with data-driven oocyte quality assessment provides clinicians with evidence-based tools for more accurate, individualized patient counseling while maintaining transparent limitations appropriate for clinical implementation~\cite{fda2022clinical}.

The parametric calculator component delivers immediate clinical value through age-dependent AMH interpretation and transparent oocyte yield predictions~\cite{lee2017amh,song2021amh}. By avoiding misleading fixed AMH reference ranges and properly incorporating age-specific percentiles, the tool enables more accurate patient counseling and prognosis discussions. The transparent mathematical relationships allow clinicians to explain prediction rationale to patients, supporting informed consent and shared decision-making processes~\cite{beauchamp2019principles,asrm2021counselors}.

The oocyte quality assessment model demonstrates realistic performance on genuine clinical data, achieving moderate but clinically meaningful correlation (r = 0.421) with blastulation outcomes~\cite{varoquaux2022machine}. The model's high sensitivity (97.6\%) provides valuable clinical utility by minimizing false negatives—an appropriate conservative approach for reproductive medicine applications~\cite{cutting2008elective}. Rigorous statistical validation through cross-validation and label-shuffled controls confirms genuine predictive capability beyond random chance~\cite{cohen1988statistical,mann1947test}.

This work establishes important precedents for responsible AI implementation in reproductive medicine by prioritizing honest performance reporting over inflated accuracy claims~\cite{rudin2019stop,topol2019high}. The emphasis on uncertainty quantification, statistical validation, and transparent limitations provides a framework for developing clinically appropriate AI tools that enhance rather than replace clinical judgment~\cite{rajkomar2019machine}.

The integrated approach offers meaningful improvements over current subjective assessment methods~\cite{paternot2009observer,paternot2011multicentre} while acknowledging the inherent complexity of predicting biological outcomes. Performance metrics reflect realistic capabilities on real clinical data, avoiding unrealistic expectations while demonstrating clinically relevant enhancements to both cycle prediction accuracy and oocyte quality assessment consistency.

Future clinical translation requires multi-center validation studies, appropriate regulatory oversight, and comprehensive clinician training programs~\cite{fda2021ai,fda2022clinical,varoquaux2022machine}. However, the framework provides a solid foundation for evidence-based personalized IVF counseling that can improve patient care through more accurate, individualized prognosis discussions and objective oocyte quality assessment support.

By combining transparent parametric modeling with sophisticated machine learning techniques, this framework demonstrates the potential for responsible AI implementation in reproductive medicine while setting important standards for honest performance reporting and clinical appropriateness in medical AI development~\cite{topol2019high,litjens2017survey}. 